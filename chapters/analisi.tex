%!TEX root = ../document.tex

Després de compilar la llibrería \textbf{FFTW} i el programa principal (\textbf{3D\_Dock}) ens disposem a analitzar el comportament d'aquest a partir de \emph{profiling}.

La primera conclusió que treiem és immediata: hi ha dos grans funcions que ocupen juntes la major part de l'\emph{elapsed time}:

\begin{itemize}
  \item \codi{electric\_field} 48.3 \%
  \item \codi{pythagoras} 26.5 \%
\end{itemize}

A continuació es pot observar com aquesta distribució marca condundentment el nostre principal objectiu a optimitzar.

\imatgepetita{img/profiling.png}{Distribució del l'elapsed time en crides a funcions}

Observant el codi, s'aprecia que la funció \codi{pythagoras} es crida des d'\codi{electric\_field}, per tant tenim que el 75\% del temps d'execució s'ocupa calculant el camp electric en cada punt d'aquesta graella.

El codi de la funció \codi{electric\_field} és prou senzill com per mostrar-lo (simplificat) a continuació:

\begin{lstlisting}[label=ef0, caption=Codi original d'electric\_field, language=C]
/* Inicialitzacions */

for( x = 0 ; x < grid_size ; x ++ )
  for( y = 0 ; y < grid_size ; y ++ )
    for( z = 0 ; z < grid_size ; z ++ )
    {
      phi = 0 ;
      for(residue = 1 ; residue <= This_Structure.length; residue++)
      {
        for(atom=1; atom <= This_Structure.Residue[residue].size; atom++)
        {
          if(This_Structure.Residue[residue].Atom[atom].charge != 0)
          {
            distance = pythagoras(
              This_Structure.Residue[residue].Atom[atom].coord[1],
              This_Structure.Residue[residue].Atom[atom].coord[2],
              This_Structure.Residue[residue].Atom[atom].coord[3],
              x_centre , y_centre , z_centre
            ) ;

            if( distance < 2.0 ) distance = 2.0 ;
            if( distance >= 2.0 )
              if( distance >= 8.0 )
                epsilon = 80 ;
              else
                if( distance <= 6.0 )
                  epsilon = 4 ;
                else
                  epsilon = ( 38 * distance ) - 224 ;

              phi += (
                This_Structure.Residue[residue].Atom[atom].charge /
                (epsilon * distance)
              );
            }
          }
        }

      }
      grid[gaddress(x,y,z,grid_size)] = (fftw_real)phi;
    }
\end{lstlisting}

El significat d'aquest codi és el següent:

\begin{itemize}
  \item Per cada punt de la graella tridimensional:
  \item Recorrem totes les mol·lècules que hem llegit i per cadascuna d'elles:
  \item Recorrem els àtoms d'aquesta, i a cada àtom:
  \item En comprovem la càrrega. Si no és nul·la:
  \item Calculem la distància entre l'àtom i el punt de la graella on estavem
  \item Calculem el camp elèctric provocat per l'àtom en aquest punt
  \item Finalment acumulem aquest càlcul final a la matriu de camps elèctrics
\end{itemize}