%!TEX root = ../document.tex

Després de compilar la llibrería \textbf{FFTW} i el programa principal (\textbf{3D\_Dock}) ens disposem a analitzar el comportament d'aquest a partir de \emph{profiling}.

\section{Profiling}

La primera conclusió que treiem és immediata: hi ha dos grans funcions que ocupen juntes la major part de l'\emph{elapsed time}:

\begin{itemize}
  \item \codi{electric\_field} 48.3 \%
  \item \codi{pythagoras} 26.5 \%
\end{itemize}

A continuació es pot observar com aquesta distribució marca condundentment el nostre principal objectiu a optimitzar.

\imatgepetita{img/profiling.png}{Distribució del l'elapsed time en crides a funcions}

Observant el codi, s'aprecia que la funció \codi{pythagoras} es crida des d'\codi{electric\_field}, per tant tenim que el 75\% del temps d'execució s'ocupa calculant el camp electric en cada punt d'aquesta graella.

\section{Anàlisi}

El codi de la funció \codi{electric\_field} és prou senzill com per mostrar-lo (simplificat) a continuació:

\begin{lstlisting}[label=ef0, caption=Codi original d'electric\_field, language=C]
/* Inicialitzacions */

for( x = 0 ; x < grid_size ; x ++ )
{
  for( y = 0 ; y < grid_size ; y ++ )
  {
    for( z = 0 ; z < grid_size ; z ++ )
    {
      phi = 0 ;
      for(residue = 1 ; residue <= This_Structure.length; residue++)
      {
        for(atom=1; atom <= This_Structure.Residue[residue].size; atom++)
        {
          if(This_Structure.Residue[residue].Atom[atom].charge != 0)
          {
            distance = pythagoras(
              This_Structure.Residue[residue].Atom[atom].coord[1],
              This_Structure.Residue[residue].Atom[atom].coord[2],
              This_Structure.Residue[residue].Atom[atom].coord[3],
              x_centre , y_centre , z_centre
            ) ;

            if( distance < 2.0 ) distance = 2.0 ;
            if( distance >= 2.0 )
            {
              if( distance >= 8.0 )
                epsilon = 80 ;
              else
                if( distance <= 6.0 )
                {
                  epsilon = 4 ;
                }
                else
                {
                  epsilon = ( 38 * distance ) - 224 ;
                }
              phi += (This_Structure.Residue[residue].Atom[atom].charge / (epsilon * distance));
            }
          }
        }
      }
      grid[gaddress(x,y,z,grid_size)] = (fftw_real)phi;
    }
  }
}
\end{lstlisting}

El significat d'aquest codi és el següent:

\begin{itemize}
  \item Per cada punt de la graella tridimensional (línia 7):
  \item Recorrem totes les mol·lècules que hem llegit i per cadascuna d'elles (10):
  \item Recorrem els àtoms d'aquesta, i a cada àtom (12):
  \item En comprovem la càrrega. Si no és nul·la (14):
  \item Calculem la distància entre l'àtom i el punt de la graella on estavem (16)
  \item Calculem el camp elèctric provocat per l'àtom en aquest punt (37)
  \item Finalment acumulem aquest càlcul final a la matriu de camps elèctrics (42)
\end{itemize}

\section{Temps d'execució}

Com hem comentat prèviament, es mesura només el temps d'execució real entre dos punts del programa. Aquests venien pre-fixats en l'enunciat. Per mesurar el temps d'execució prenem dos captures de l'instant del rellotge del sistema, per tant mesurem la diferència de temps entre els dos instants amb precisió de $\mu$ segons:

\begin{lstlisting}[label=timer, caption=Mesura del temps d'execució real, language=C]
struct timeval start, end;
gettimeofday(&start, NULL);

/* ... timed code ... */

gettimeofday(&end, NULL);
float elapsed = (end.tv_sec + (end.tv_usec / 1000000.0)) -
                (start.tv_sec + (start.tv_usec / 1000000.0));
fprintf(stderr, "%f\n", elapsed);
\end{lstlisting}

Amb això tenim que el programa treurà pel canal 2 (sortida d'error estàndard) l'\emph{elapsed time}. El nostre script d'executar el programa llegirà aquest valor per calcular-ne el temps promig.

En una primera execució del programa, sense haver fet encara cap modificació excepte l'instrumentació per capturar el temps d'execució real entre els dos punts esmentats, obtenim els següents resultats:


\begin{figure}[ht]
  \caption{Temps d'execució del programa original - segons}\label{fig:elapsed_orig}
  \begin{center}
    \begin{tabular}{ l r r r r }
      & test1 & test2 & test3 & total \\
      \hline
      Versió original & 45.5 & 174.5 & 164.3 & 384.3 \\
    \end{tabular}
  \end{center}
\end{figure}

Cada execució que es mostri a partir d'ara, a l'igual que aquesta, serà el resultat d'executar per a una versió específica del programa amb els tres jocs de proves proporcionats i iterant 5 vegades cada prova. Els valors mostrats són les mitjanes d'aquestes execucions. Quan parlem de guany, sempre ho farem respecte a aquests temps d'execució.
