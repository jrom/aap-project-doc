%!TEX root = ../document.tex

\subsection{Unroll i vectorització de sqrt}

Un cop vectoritzades les operacions auxiliars del cos de \codi{pythagoras}, i aprofitant el separament de bucles fet anteriorment, apliquem un unrolling de grau 4 per tal de poder calcular paral·lelament mitjançant operacions vectorials l'operació d'arrel quadrada que és molt costosa.

El bucle que hem mostrat a l'apartat anterior acaba amb aquest aspecte:

\begin{lstlisting}[label=vect2, caption=Unrolling i vectorització de pythagoras, language=C]
aux_distance = atom_distances;
__m128 _centers = _mm_setr_ps(x_centre, y_centre, z_centre, 0.0);
for ( atom = 0; atom < atoms-3; atom += 4)
{
  __m128 pyths;
  __m128 aux = _mm_sub_ps(*((__m128*) aux_coord), _centers);
  aux = _mm_mul_ps(aux, aux);
  *((float*) &pyths) = *((float*) &aux) + *((float*) (&aux)+1) + *((float*) (&aux)+2);
  aux_coord += 4;

  aux = _mm_sub_ps(*((__m128*) aux_coord), _centers);
  aux = _mm_mul_ps(aux, aux);
  *((float*) &pyths+1) = *((float*) &aux) + *((float*) (&aux)+1) + *((float*) (&aux)+2);
  aux_coord += 4;

  aux = _mm_sub_ps(*((__m128*) aux_coord), _centers);
  aux = _mm_mul_ps(aux, aux);
  *((float*) &pyths+2) = *((float*) &aux) + *((float*) (&aux)+1) + *((float*) (&aux)+2);
  aux_coord += 4;

  aux = _mm_sub_ps(*((__m128*) aux_coord), _centers);
  aux = _mm_mul_ps(aux, aux);
  *((float*) &pyths+3) = *((float*) &aux) + *((float*) (&aux)+1) + *((float*) (&aux)+2);
  aux_coord += 4;

  *((__m128*) aux_distance) = _mm_sqrt_ps(pyths);
  aux_distance +=4 ;
}

for ( ; atom < atoms; atom++)
{
  __m128 aux = _mm_sub_ps(*((__m128*) aux_coord), _centers);
  aux = _mm_mul_ps(aux, aux);
  *aux_distance = sqrt( *((float*) &aux) + *((float*) (&aux)+1) + *((float*) (&aux)+2) );
  aux_coord += 4; aux_distance++;
}
\end{lstlisting}

Com es pot observar, l'unroll crea la necessitat d'un segon bucle per assegurar que quan \codi{atoms} no és multiple de 4 també es processin tots els àtoms. El que s'ha canviat és que ara es processen quatre àtoms, i finalment la operació d'arrel quadrada s'aplica a les distàncies dels quatre àtoms alhora mitjançant l'operació vectorial \codi{\_mm\_sqrt\_ps}. La resta segueix igual. Aquest canvi ens deixa amb els següents resultats:

\begin{figure}[ht]
  \caption{Resultats amb una millor organització de les dades en memòria}\label{fig:elapsed_3}
  \begin{center}
    \begin{tabular}{ l r r r r }
      & test1 & test2 & test3 & total \\
      \hline
      Versió original & 45.5 & 174.5 & 164.3 & 384.3 \\
      Optimització 1 & 31.2 & 119.0 & 104.9 & 255.1 \\
      Optimització 2 & 27.8 & 107.4 & 91.6 & 226.8 \\
      \textbf{Optimització 3} & 22.2 & 84.2 & 68.3 & 174.7 \\
    \end{tabular}
  \end{center}
\end{figure}

Amb aquestes millores el nostre programa ja funcioni més del doble de ràpid que l'original, amb un guany global de \textbf{2.2} i en el cas millor de \textbf{2.41}.
